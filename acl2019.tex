%
% File acl2019.tex
%
%% Based on the style files for ACL 2018, NAACL 2018/19, which were
%% Based on the style files for ACL-2015, with some improvements
%%  taken from the NAACL-2016 style
%% Based on the style files for ACL-2014, which were, in turn,
%% based on ACL-2013, ACL-2012, ACL-2011, ACL-2010, ACL-IJCNLP-2009,
%% EACL-2009, IJCNLP-2008...
%% Based on the style files for EACL 2006 by 
%%e.agirre@ehu.es or Sergi.Balari@uab.es
%% and that of ACL 08 by Joakim Nivre and Noah Smith

\documentclass[11pt,a4paper]{article}
\usepackage[hyperref]{acl2019}
\usepackage{times}
\usepackage{latexsym}

\usepackage{url}

%\aclfinalcopy % Uncomment this line for the final submission
%\def\aclpaperid{***} %  Enter the acl Paper ID here

%\setlength\titlebox{5cm}
% You can expand the titlebox if you need extra space
% to show all the authors. Please do not make the titlebox
% smaller than 5cm (the original size); we will check this
% in the camera-ready version and ask you to change it back.

\newcommand\BibTeX{B\textsc{ib}\TeX}

\title{Instructions for ACL 2019 Proceedings}

\author{First Author \\
  Affiliation / Address line 1 \\
  Affiliation / Address line 2 \\
  Affiliation / Address line 3 \\
  \texttt{email@domain} \\\And
  Second Author \\
  Affiliation / Address line 1 \\
  Affiliation / Address line 2 \\
  Affiliation / Address line 3 \\
  \texttt{email@domain} \\}

\date{}

\begin{document}
\maketitle
\begin{abstract}
  % Automatic summarization research has made substantial progress thanks to novel methods and datasets. Manual evaluation approaches so far either ignore content and focus on fluency, or require expert annotators but nevertheless suffer from low inter-annotator agreement due to the complexity of the task. In the few cases where the contents of the summary are evaluated, the evaluation is biased due to using a single reference summary, which results in different summaries of equal quality being rated according to their similarity to the reference. In this paper, we propose a Highlight-bAsed Evaluation of Single document Summarization (HArnESS). Our proposal assesses summaries against the original document, facilitated through manually highlighted salient content which can be reused in future studies. Furthermore it does not require expert annotators, avoids reference bias and provides absolute instead of ranked evaluation of systems.
  Automatic summarization research has made substantial progress thanks to novel methods and datasets. 
  %OLD:
  %However, a well-accepted manual evaluation for content such as Pyramid requires expert annotation which is often not available for many datasets. As such, current practices choose to assess the content by directly comparing the summary against the original document or a reference summary. 
  %Our finding found that these practices are not suitable for long document or dataset that only provides a single reference summary per document as these cases are often result in scores that have high variability between judges' assessment due to bias. 
  %AV: I would replace the 3 sentences above with something like: 
  Despite this progress, most manual evaluation focuses on fluency but not on the content of the summaries. Those that do, typically use a single reference summary for comparison, either directly or through questions, which introduces a bias towards a single correct answer for a task where this assumption doesn't hold.
  To address this issue, we propose Highlight-bAsed Evaluation of Single document Summarization (HArnESS) that provides assessment of a summary against the original document, facilitated through manually highlighted salient content. The highlights lower the variability of the judges' assessment and are reusable in future studies. Furthermore it does not require expert annotators, avoids reference bias and provides absolute instead of ranked evaluation of the systems.
  
\end{abstract}

\section{Introduction}
\cite{Celikyilmaz2018}


\section*{Acknowledgments}

The acknowledgments should go immediately before the references.  Do
not number the acknowledgments section. Do not include this section
when submitting your paper for review. \\

\noindent \textbf{Preparing References:} \\
Include your own bib file like this:
\verb|\bibliographystyle{acl_natbib}|
\verb|\bibliography{acl2019}| 

where \verb|acl2019| corresponds to a acl2019.bib file.
\bibliography{acl2019}
\bibliographystyle{acl_natbib}

\appendix

\section{Appendices}
\label{sec:appendix}
Appendices are material that can be read, and include lemmas, formulas, proofs, and tables that are not critical to the reading and understanding of the paper. 
Appendices should be \textbf{uploaded as supplementary material} when submitting the paper for review. Upon acceptance, the appendices come after the references, as shown here. Use
\verb|\appendix| before any appendix section to switch the section
numbering over to letters.


\section{Supplemental Material}
\label{sec:supplemental}
Submissions may include non-readable supplementary material used in the work and described in the paper. Any accompanying software and/or data should include licenses and documentation of research review as appropriate. Supplementary material may report preprocessing decisions, model parameters, and other details necessary for the replication of the experiments reported in the paper. Seemingly small preprocessing decisions can sometimes make a large difference in performance, so it is crucial to record such decisions to precisely characterize state-of-the-art methods. 

Nonetheless, supplementary material should be supplementary (rather
than central) to the paper. \textbf{Submissions that misuse the supplementary 
material may be rejected without review.}
Supplementary material may include explanations or details
of proofs or derivations that do not fit into the paper, lists of
features or feature templates, sample inputs and outputs for a system,
pseudo-code or source code, and data. (Source code and data should
be separate uploads, rather than part of the paper).

The paper should not rely on the supplementary material: while the paper
may refer to and cite the supplementary material and the supplementary material will be available to the
reviewers, they will not be asked to review the
supplementary material.


\end{document}
